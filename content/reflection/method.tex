%!TEX root = ../../super_main.tex

\section{Scrum in \giraf}
% Diskuter hvorfor Scrum måske på nogle måder ikke giver mening i context af giraf pga vidensdeling mellem semestre

Towards the end of the project, we got to reflect a little on the \giraf development method. We found the that the overall agile approach, without much documentation besides the reports, used by most groups this semester might cause some trouble in the future, because of the lack of written documentation to support knowledge transfer between semesters.
\\\\
One might say that the semester reports provide such documentation, but in reality the reports are too scattered across the different semesters and the different projects to be useful. Old semester reports might also include information that is no longer relevant to the next semesters. 
\\\\
\todo{TODO Start - Niklas}
The transition process between semesters could possibly run much smoother if every subproject had its own documentation to support knowledge transfer. This documentation could then iteratively be further developed each semester. The agile Scrum approach during this semester worked well in terms of development, but we suggest future semester groups to consider adding a written documentation as a work product towards the end of their semester to properly support knowledge transfer between semesters. We have tried to avoid loading our report with too much required information and instead use tools such as Redmine and other documents such as the Android guidelines and the design manual.

\subsection{Scrum Meetings} \todo{Broad split -- subsection? - NOTE: Det er lidt mærkeligt med en subsection her eftersom det er den eneste der er i afsnittet.. Overvej om den skal væk igen }
At the beginning of the semester, we decided upon holding two weekly Scrum meetings in every subproject (see \secref{cha:introduction}), and this was held during the first 2 sprints with relative success. Thereafter, once the other groups started having a lot to do, this slowly decreased to once per week or once every second week. In our group, we felt that the person who was responsible for arranging the meetings did not live up to his responsibilities. \todo{Since we were a major contributing part etc.} We do however feel that our own group did not have a need for more meetings, while the other groups might have had a bigger need for it. This may due to the fact that we are the creators of the shared components and thus not as dependent on other groups as they are on us. However, we cannot know for sure if having these meetings would have actually mattered for our group. \todo{hvorfor har de haft mere brug for det, knap så meget pik inde i girafcomponents. Note: Jeg har skrevet at vi ikke kan være sikre på om møderne har haft nogen betydning - er det nok?}
\\\\ \todo{overvej om at møder kunne have hjulpet anyway, ..you dont know what you dont know..}
We therefore strongly encourage future students to properly manage their meetings and communicate both in their own group, with other groups in their subproject, and make sure that knowledge is shared across the entire project. \todo{Skriv at vi har forsøgt at starte en sådan knowlegde base (design manualen) NOTE: Det her er skrevet lidt længere oppe}

\section{Collaboration}
Throughout the different sprints we have been working with a lot of different groups in different parts of the three sub projects. Overall, we feel like the collaboration we did with other groups (both undocumented and documented) have been useful by given us an insight in a more ``real'' development environment.
\\\\
With this said, we still feel like the communication between sub projects could have been better. For instance, communication with the database sub project have been very hard. While developing on the \ct, we requested some features which could be implemented relatively easily (accordingly to the database groups). However, whenever we asked the group responsible a progress update the answer was always ``soon'', ``tomorrow'' or something like that. After a while, we got to the point where we would just implement fixes and maybe change the implementation whenever the change had been made in the database projects. 
\todo{Reflekter over samarbejde med andre grupper}

\section{Group Synergy}
\todo[inline]{Niklas note: Jeg har lidt svært med at komme igang med det her afsnit. Jeg tænker at der skal stå noget i retningen af: Vi har arbejdet godt og effektivt. Måske kan det være fordi vi har arbejdet sammen før og kender hinanden udenfor skolen. Desuden har vi været gode til at motivere hinanden og hjælpe hinanden hvor der har været problemer}
\todo{Reflekter over arbejde internt i gruppen}

\todo{End Start - Niklas}