%!TEX root = ../../super_main.tex

\section{Scrum in \giraf}
% Diskuter hvorfor Scrum måske på nogle måder ikke giver mening i context af giraf pga vidensdeling mellem semestre

Towards the end of the project, we got to reflect a little on the \giraf development method. We found the that the overall agile approach, without much documentation besides the reports, used by most groups this semester might cause some trouble in the future, because of the lack of written documentation to support knowledge transfer between semesters.
\\\\
One might say that the semester reports provide such documentation, but in reality the reports are too scattered across the different semesters and the different projects to be useful. Old semester reports might also include information that is no longer relevant to the next semesters. 
\\\\
The transition process between semesters could possibly run much smoother if every subproject had its own documentation to support knowledge transfer. This documentation could then iteratively be further developed each semester. The agile Scrum approach during this semester worked well in terms of development, but we suggest future semester groups to consider adding a written documentation as a work product towards the end of their semester to properly support knowledge transfer between semesters. We have tried to avoid loading our report with too much required information and instead use tools such as Redmine and other documents such as the Android guidelines and the design manual.

\subsection{Scrum Meetings}
At the beginning of the semester, we decided upon holding two weekly Scrum meetings in every subproject (see \secref{cha:introduction}), and this was held during the first 2 sprints with relative success. Thereafter, once the other groups started having a lot to do, this slowly decreased to once per week or once every second week. In our group, we felt that the person who was responsible for arranging the meetings did not live up to his responsibilities. We do however feel that our own group did not have a need for more meetings, while the other groups might have had a bigger need for it. This may be due to the fact that we are the creators of most of the shared components, as well as a major contributing part to the design decisions, and thus not as dependent on other groups as they are on us. However, we cannot know for sure if having these meetings would actually have mattered for our group.
\\\\ 
We therefore strongly encourage future students to properly manage their meetings and communicate both in their own group, with other groups in their subproject, and make sure that knowledge is shared across the entire project.

\section{Collaboration}
Throughout the different sprints we have been working with a lot of different groups in different parts of the three sub projects. Overall, we feel like the collaboration we did with other groups (both undocumented and documented) have been useful by giving us insight in a more ``real'' development environment.
\\\\
With this said, we still feel like the communication between sub projects could have been better. For instance, communication with the database sub project has been very hard. While developing on the \ct, we requested some features which could be implemented relatively easily (according to the database groups). However, whenever we asked the group responsible about a progress update, the answer was always ``soon'', ``tomorrow'' or something like that. After a while, we got to the point where we would just implement workarounds and maybe change the implementation whenever the requested feature had been implemented in the database projects. 

\section{Our development process}
The groups were encouraged to use Scrum in the \giraf Development method for managing the development process this semester. We started out by following Scrum strictly with daily stand-up meetings. However, over time we gradually stopped following Scrum and we stopped having daily stand-up meetings. However because the group only consisted of four members the ceremony and structure of Scrum did not seem necessary internally in the group. For instance knowledge sharing was simply done ad hoc instead of using daily stand up meetings. However, the group, as a unit, still followed the Scrum process on a multi project scale.

