%!TEX root = ../../super_main.tex

\section{Scrum in \giraf}
% Diskuter hvorfor Scrum måske på nogle måder ikke giver mening i context af giraf pga vidensdeling mellem semestre

Towards the end of the project, we got to reflect a little on the \giraf development method. We found the that the overall agile approach, without much documentation besides the reports, used by most groups this semester might cause some trouble in the future, because of the lack of written documentation to support knowledge transfer between semesters.
\\\\
One might say that the semester reports provide such documentation, but in reality the reports are too scattered across the different semesters and the different projects to be useful. Old semester reports might also include information that is no longer relevant to the next semesters. 
\\\\
The transition process between semesters could possibly run much smoother if every subproject had its own documentation to support knowledge transfer. This documentation could then iteratively be further developed each semester. The agile Scrum approach during this semester worked well in terms of development, but we suggest future semester groups to consider adding a written documentation as a work product towards the end of their semester to properly support knowledge transfer between semesters.
\\\\ \todo{Broad split -- subsection?}
At the beginning of the semester, we decided upon holding two weekly Scrum meetings in every subproject (see \secref{cha:introduction}), and this was held during the first 2 sprints with relative success. Thereafter, once the other groups started having a lot to do, this slowly decreased to once per week or once every second week. In our group, we felt that the person who was responsible for arranging the meetings did not live up to his responsibilities. \todo{Since we were a major contributing part etc.} We do however feel that our own group did not have a need for more meetings, while the other groups might have had a bigger need for it \todo{hvorfor har de haft mere brug for det, knap så meget pik inde i girafcomponents}.
\\\\ \todo{overvej om at møder kunne have hjulpet anyway, ..you dont know what you dont know..}
We therefore strongly encourage future students to properly manage their meetings and communicate both in their own group, with other groups in their subproject, and make sure that knowledge is shared across the entire project. \todo{Skriv at vi har forsøgt at starte en sådan knowlegde base (design manualen)}

\todo{Reflekter over samarbejde med andre grupper}

\todo{Reflekter over arbejde internt i gruppen}