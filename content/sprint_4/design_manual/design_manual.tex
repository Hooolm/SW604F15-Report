%!TEX root = ../../../super_main.tex

\chapter{Design Manual}
\label{cha:design_manual}

Throughout the fourth sprint development of the design manual had a high priority. Roughly half of the groups resources were used on the development of the manual. From the start, we knew that we would not be able to complete the manual, so we focused a lot on getting already agreed-upon design guides formalized. This chapter will describe the development of the manual. The latest version of the design manual can be found in \appref{app:design_manual}.

\section{Development}
\label{sec:development}
In the third sprint a GUI design-meeting was arranged. The focus of this meeting was to agree upon the general design of applications in \giraf. The results of this meeting were meant to be implemented in the former design guidelines made by group SW606F15. However, this was never done on more than a note basis and would not be suitable for teaching new student about the design. Because of this, it was prioritized to get these notes formalized and described in details in the new design manual. 

\section{Ensuring That Applications Comply with the Manual}
\label{sec:ensuring_that_applications_comply_with_the_manual}
While developing the manual, we looked at the different applications in the \giraf software suite to see if we could help other groups with improving the design of their applications so that their applications could become more usable and more consistent with the other applications. After checking the different applications, we sat down with members from the various groups and gave them a walk-through of the found errors such that they had time to correct them before the Sprint Review Meeting. 
\\\\
While looking through the different applications we found some design cases that were not considered in the original design guidelines. Due to limited time in the fourth sprint, it was decided to not write additional sections in the manual to cover these cases. Instead of just deciding upon the design in these cases for ourself, we have written notes throughout the design-manual to indicate where features are missing. These notes describe the different choices that the next students will have to consider in order to complete the manual. Ideally the groups next semester should have a meeting between all GUI-groups to discuss how these things should be described.