%!TEX root = ../../super_main.tex
\chapter{Introduction}
\label{cha:introduction}

\todo[inline]{Insert mild intro. Techonology helps our lives get better -> Folk with autism can be helped}

% Purpose of software
This report focuses on the continued development of an existing \emph{Android}-software suite for tablets called \giraf(Graphical Interface Resources for Autistic Folk), which includes tools and games to assist and train everyday aspects of life for citizens diagnosed an Autism Spectrum Disorder (ASD) and their guardians. 

One motivation for this tool is to digitize some of the existing tools that people with an autism disorder, already uses. 

One of the main stakeholder of the \giraf-system is a kindergarten called \emph{Birken} for children diagnosed with an ASD. In this kindergarten the staff spends a lot of time preparing visual schedules and checklists for the children which helps with their daily functions. 
 %for them to be better to understand basic social strategies. 
This could be for instance be a schedule for the day for a child, constructed of pictograms, simple pictures, laminated and applied to different surfaces using Velcro-tape for the children to inspect.

\begin{table}[!htbp]
    \begin{tabular}{|l|l|l|}
    \hline
    \textbf{Role}				& \textbf{Group}  & \textbf{Description}                                   \\ \hline
    \multirow{2}{*}{Users}      & Citizen         & Citizens diagnosed with ASD                            \\ \cline{2-3}
             					& Guardian        & Institutional guardians of citizens diagnosed with ASD \\ \hline
    Costumers 					& Social services & The municipality of Aalborg                            \\ \hline
    \end{tabular}
\end{table}
\todo[inline]{Refer to this table and add caption and label}



