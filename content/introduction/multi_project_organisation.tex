%!TEX root = ../../super_main.tex

\section{Multi Project Organization}
\label{sec:multi_project_organization}

This semester project is the continuation of a widely spanning multi project with more than 50 developers which are organized in roughly 15 smaller groups, up to four developers per group, including a group consisting of the authors of this report. The multi project was split into three different project areas: \emph{Build and Deployment}, \emph{GUI}, \emph{Database}. \todo{Describe the different project groups}

\subsection{Organization}
The multi project was organized using the process management method, Scrum \parencite{scrum}, more specific Scrum of Scrums. This method allows the groups in the multiproject to be self-organized, and enforces the individual groups to do work more independently. This independent work method fits great in since the individual groups are used to work in smaller projects groups. During the semester, the groups will acquire experience in working on a larger scale projects than previously encountered. \todo{Reflect (in the reflection section) upon the stand up meetings that werent really held. Maybe we should've held some actual stand up meeting alongside normal meetings}
\\\\
As seen in \figref{fig:scrum_of_scrums} the Scrum of Scrums is split into three levels. The bottom most level is the individual groups, which work exactly as an ordinary development group using the Scrum method \todo{This sentence is incorrect. The individual groups decide for themselves. Do not write ordinary <-- marhlderfix}. The middle layer contains the three project areas where representatives from each individual group meet and synchronize their work at least twice a week. The top layer in \figref{fig:scrum_of_scrums} contains representatives from all of the project areas. Since stand-up meetings only occur once a week, it is important that every group sends a Scrum representative to join the meetings.

\begin{figure}[!htbp]
  \centering
    \includegraphics[width=0.8\textwidth]{scrum_of_scrums}
    \caption{Organization in Scrum of Scrums and meeting frequencies}
    \label{fig:scrum_of_scrums}
\end{figure}

\subsection{Roles and Responsibilities}
Several different roles and responsibilities were divided among the project groups. This group volunteered as ``Webadmin'' for the website at \url{http://giraf.cs.aau.dk/}. The ``Webadmin'' responsibility included administration of the official \giraf web page. Because one of the members had previous experience with Android, he was appointed as ``Android Guru''. His responsibility included helping out with Android related issues that groups could not handle within reasonable time themselves and setting up a general Android guideline. \todo{Reflect upon this! Write that we did not really do anything (the only user story was to change the favicon). Besides this, ulrik did errthing}

\subsection{Further Reading}
For further description of the \giraf Development method we refer to the description made by group SW609F15 in their report\footnote{This report has not yet been published, and we are therefore unable to cite it properly}.