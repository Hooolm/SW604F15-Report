%!TEX root = ../../super_main.tex
\chapter{Semester Conclusion}
\label{cha:conclusion_final}

The reader is invited to reference \appref{app:screenshots_of_the_finished_applications}, containing screenshots of the final version of the developed applications and components, while reading this chapter. The applications can also be downloaded from the Google Play store \parencite{giraf_google_play}.

\section*{\launcher}
The \launcher is the main entry point to the \giraf system and our group have been responsible for maintaining and further developing the application throughout the semester. The \launcher was found to be unstable in the early development. After four sprints of development, our group have brought significant improvements to the \launcher, in terms of both stability and usability.
\\\\
The overall stability has been improved by for instance properly utilizing fragments in the launcher settings and making sure that the GUI thread is kept free from heavy processing. 
\\\\
The usability has been improved by streamlining the design and by optimizing features such as the newly implemented view pager in the home screen of the \launcher.

\section*{\emph{Category Tool}}
The \ct was not able to compile when it was assigned to our group. When the \ct was brought to a running state it was found that the design was highly inconsistent and did not follow basic design principles. The application was in such a bad condition that a redesign was found to be the best approach in comparison to repairing it.
\\\\
Avoiding the discovered pitfalls along with following the Android guidelines, had a high priority in redesigning the \ct. This included splitting the functionality into logically grouped screens. For instance viewing categories and creating categories was split into two different activities.
\\\\
We did not want to add unused features from the previous version in the new design, such as coloring categories. Therefore, during the redesign, the relevant user stories for the \ct was found. These user stories included but were not limited to general management of categories, e.g. creating and removing categories alongside populating them with pictograms. These user stories were then implemented and unused features were avoided. 
\\\\
Furthermore, we found new user stories for \ct such as templates of categories that can be copied to citizens and then later customized further. The found user stories was all implemented and satisfied the customers. Usability evaluation indicated the \ct worked as intended with only a few cosmetic problems.

\section*{\giraf \emph{Components} and Design Manual}
In the early stage of the development the new \gc was introduced in order to streamline the design of the \giraf applications. However, since we got the responsibility for implementing this shared design it motivated us to create a knowledge base for the development such that our work and choices regarding design would not be lost for the next semester. When we started developing there was no shared knowledge base assisting us on how we should approach the development. When creating the design manual and the shared components we wanted them to work nicely together such that it would be easier to maintain and further develop. This is perhaps our most significant contribution to the overall project.