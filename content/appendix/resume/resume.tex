%!TEX root = ../../../super_main.tex
\chapter{Resumé}
\giraf Autism Application Suite is an Android Tablet Application suite which helps people diagnosed with Autism Spectrum Disorder (ASD) with different aspects of their every day lives. Many of ideas and principles used in \giraf are already used in different institutions that takes care of people diagnosed with ASD. An example of such a tool could be a sequence of printed laminated pictograms used to show how to perform an every day task.       
\\\\
\giraf is developed in a multi-project setting organized as Scrum of Scrums with the lowest level of Scrums being individual bachelor project groups. The process included meetings with costumers at the end of each sprint. Collaboration between semester groups is key to the success of \giraf we have been actively working together with different groups, mostly groups also working on Android applications.  
\\\\
The focus throughout this semester has been to unify and stabilize the different \giraf Android Applications. This included resolving bugs and trying to make all the applications appear and work as a single package. The overall goal was to make to applications stable enough to be used in their intended environment as most features were already or partially complete.   
\\\\
We have contributed to the project with an improved and stabilized version of the \giraf \launcher application and a complete rework of the existing \giraf \ct. Our, perhaps, greatest contribution has been new and reworked components in the \gc library to help streamlining the different applications. We have contributed to \gc with a series of different standard components including, but not limited to: standardized dialogs, a standard button, a standard way of displaying pictograms, a standard top bar, and a bug splat screen to display error messages.
\\\\
We have also started and filled in a design manual, to be used as knowledge transfer medium between semesters, which we expect future \giraf developers to continue building upon. The design manual includes information about design decisions throughout \giraf.  
\\\\
From the usability evaluations, conducted by others, we have found that our contributions were accepted by the customers. A series of unstructured pseudo random touch event input tests (Monkey Tests) have shown our applications to be resilient and stable. The applications were presented at the end of the semester to the customers and we had the newest versions of the applications 
