%!TEX root = ../../../super_main.tex
\chapter{Resumé}
\todo[inline]{DONE-start (Mathias) Har læst marhlders ting igennem og rettet ting. kig på det}
\giraf Autism Application Suite is an Android Tablet Application suite which helps people diagnosed with Autism Spectrum Disorder (ASD) with different aspects of their every day lives. Many of the ideas and principles used in \giraf are already used in different institutions that take care of people diagnosed with ASD. An example of such a tool could be a sequence of printed laminated pictograms used to show how to perform an every day task.
\\\\
The focus throughout this semester has been to unify and stabilize the different \giraf Android Applications. This included resolving bugs and trying to make all the applications appear and work as a single package. The overall goal was to make to applications stable enough to be used in their intended environment as most features were already fully or partially complete.  
\\\\
\giraf is developed in a multi-project setting organized as Scrum of Scrums with the lowest level of Scrums being individual bachelor project groups. Collaboration between these groups is key to the success of \giraf, as the development is a collaboration between more than 50 $6^{th}$ semester software engineering students. All these students are split into three subprojects, which include Graphical User Interface (front end), Database (back end) and Build and Deployment (application shipping, configuration management and automatic builds). Our group was part of the GUI subproject, and the main collaboration partners for our group include the other front end developers. We documented our collaboration with group SW606F15, whom had a problem with deletion of \giraf Sequences and the implementation of Android Adapters, and group SW613F15, whom had some issues with their implementation of a \androidinline{ViewPager} from the Android Support library. 
\\\\
We have contributed to the project with an improved and stabilized version of the \giraf \launcher application, used for authentication and for starting \giraf applications, and a complete rework of the existing \giraf \ct, which is used to create categories of pictograms used in other applications. Our, perhaps, greatest contribution has been new and reworked components in the \gc library to help streamlining the different applications. We have contributed to the \gc with a series of different standard components including, but not limited to: standardized dialogs, a standard button, a standard way of displaying pictograms, a standard top bar, and a bug splat screen to display error messages.
\\\\
We have also started and filled in a design manual, to be used as knowledge transfer medium between semesters, which we expect future \giraf developers to continue building upon. The design manual includes information about design decisions throughout \giraf.  
\\\\
From the usability evaluations, conducted by other software 6 groups, we have found that our contributions were accepted by the customers. Furthermore, a series of unstructured pseudo random touch event input tests (Monkey Tests) have shown our applications to be resilient and stable. The applications were presented at the end of the semester to the customers and we had the newest versions of the applications uploaded to the Google Play application store. 
\todo[inline]{DONE-end}