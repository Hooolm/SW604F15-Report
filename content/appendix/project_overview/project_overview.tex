%!TEX root = ../../../super_main.tex
\chapter{Project Overview} % (fold)
\label{cha:project_overview}

This appendix describes each of the applications and libraries that have been the responsibility of the GUI subproject in the \giraf software suite. 

\section{\emph{Launcher}}
\label{sec:app_launcher}
The \launcher is the main entry to the system, this application is used to maintain an application filter for citizens diagnosed with ASD.

\section{\emph{GIRAF Components}}
\label{sec:app_giraf_components}
This is a library which contains all the shared GUI components that should be used to streamline the design of the \giraf software suite.

\section{\emph{Pictosearch}}
\label{sec:app_pictosearch}
This is a library that allows user to search for pictograms and categories in various use cases. This application is called \emph{Piktosøger} in Danish.

\section{Week Schedule}
\label{sec:app_week_schedule}
This application is a digitized version of a board containing sequences of tasks to be performed during the week. This application is called \emph{Ugeplan} in Danish 

\section{\emph{Sequence}}
\label{sec:app_sequence}
This application is a digitized version of a board containing a sequence of a task that a citizen is trying to learn or mimic, e.g. how to get dressed or the correct process for using a toilet. This application is called \emph{Sekvens} in Danish.

\section{\emph{Picto Reader}}
\label{sec:app_picto_reader}
The citizens with no verbal language use this application as a replacement for speaking. A citizen can place a sequence of pictograms in a list, which can then be read out loud. The application is also a training tool, intended to teach the citizens how to pronounce various things. This application is called \emph{Piktooplæser} in Danish.

\section{\emph{Picto Creator}}
\label{sec:app_picto_creator}
This is a tool used by the guardians to customize pictograms such that they fit the citizens in the best way possible. The application is called \emph{Piktotegner} in Danish.

\section{\emph{Category Tool}}
\label{sec:app_category_tool}
This is a management tool for categories of pictograms. These categories can be used by other applications. This tool allows for template-categories alongside customizable categories for each citizen. This application is called \emph{Kategoriværktøjet} in Danish.

\section{\emph{Administration}}
\label{sec:app_administration}
This is another management tool, which manage the profiles and departments involved in the use of the \giraf software suite. This application currently allows for a guardian or administrator to setup profiles and relations between citizens and departments.

\section{\emph{Timer}}
\label{sec:app_timer}
This is an application that can setup a timer overlay on any other application. This teaches citizens diagnosed with ASD to understand the concept of time frames. This application is called \emph{Tidstager} in Danish.

\section{\emph{Voice Game}}
\label{sec:app_voice_game}
This is a game that teaches citizens diagnosed with ASD how to control their voice level. The game consists of a car that moves up and down depending on how loud the player speaks. By creating custom maps for this game, a guardian can help a citizen with understanding this concept in a playful way. This application is called \emph{Stemmespillet} in Danish.

\section{\emph{Category Game}}
\label{sec:app_category_game}
This is a game that teaches citizens diagnosed with ASD how to categorize objects. For instance apples and bananas belong in a food category, while socks and shirts belong in a clothes category. This game is called \emph{Kategorispillet} in Danish.

\section{\emph{Life Story}}
\label{sec:app_life_story}
This is an application that helps citizens diagnosed with ASD with explaining different events in their lives to other people. It furthermore helps them memorize such events. This application is called \emph{Livshistorier} in Danish.

\section{\emph{Web Admin}}
\label{sec:app_web_admin}
This is a web interface that allows one to manage the various parts of the \giraf system. This interface is for convenience for guardians and administrators when manipulating many items in the system, this interface supports all the features that exist in the tablet applications and more.