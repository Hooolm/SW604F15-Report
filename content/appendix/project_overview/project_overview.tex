%!TEX root = ../../../super_main.tex
\chapter{Project Overview} % (fold)
\label{cha:project_overview}

This appendix describes each of the application and libraries that have been in the responsibility of the GUI sub project in the \giraf software suite. 

\section{\emph{Launcher}}
\label{sec:app_launcher}
The \launcher is the main entry to the system, this application is used to maintain an application filter for citizens diagnosed with ASD.

\section{\emph{GIRAF Components}}
\label{sec:app_giraf_components}
This is a library contains all the shared GUI components that should be used to streamline the design of the \giraf software suite.

\section{\emph{Pictosearch}}
\label{sec:app_pictosearch}
This is a library that allows for a user to search in pictograms and categories in various use cases. This application is called \emph{Piktosøger} in Danish.

\section{Week Schedule}
\label{sec:app_week_schedule}
This application is a digitize version of a board containing sequences of tasks to be performed during the week. This application is called \emph{Ugeplan} in Danish 

\section{\emph{Sequence}}
\label{sec:app_sequence}
This application is a digitize version of a board containing a sequence of a task that at citizen is trying to learn or mimic, e.g. how to get dressed. This application is called \emph{Sekvens} in Danish.

\section{\emph{Picto Reader}}
\label{sec:app_picto_reader}
The citizens uses this application to learn verbal language by the use of pictograms and sound. A citizen place a sequence of pictograms which can then be read out loud. This application is called \emph{Piktooplæser} in Danish.

\section{\emph{Picto Creator}}
\label{sec:app_picto_creator}
This is a tool used by the guardians to customize pictograms such that they fit the citizens in the best way possible. This application is \giraf's answer to paint to edit pictograms. The application is called \emph{Piktotegner} in Danish.

\section{\emph{Category Tool}}
\label{sec:app_category_tool}
This is an managing tool that managed categories of pictograms which is used by other applications. This tool allows for templates of categories alongside customizable categories for each citizens. This application is called \emph{Kategoriværktøjet} in Danish.

\section{\emph{Administration}}
\label{sec:app_administration}
This is another managing tool that manage the profiles and departments involved in the use of the \giraf software suite. This application allows for a guardian or administrator to setup profiles and relation between citizens and departments.

\section{\emph{Timer}}
\label{sec:app_timer}
This is an application that can setup a timer overlay on any other application. This assists citizens diagnosed with ASD to understand social terms of time. This application is called \emph{Tidstager} in Danish.

\section{\emph{Voice Game}}
\label{sec:app_voice_game}
This is a game that helps citizens diagnosed with ASD to understand the social terms of sound and loudness. The game consists of a car that moves up and down depending on how the player speaks. By creating custom maps for this game, a guardian can help an individual to understand this concept in a playful way. This application is called \emph{Stemmespillet} in Danish.

\section{\emph{Category Game}}
\label{sec:app_category_game}
This is a game that helps citizens diagnosed with ASD to understand the social terms of categorization. For instance apples and bananas is food and socks and shirts are clothes. This game is called \emph{Kategorispillet} in Danish.

\section{\emph{Life Story}}
\label{sec:app_life_story}
This is an application that helps citizens diagnosed with ASD to communicate what they have done previously but it also assists them to memorize what they have done. This application is called \emph{Livshistorier} in Danish.

\section{\emph{Web Admin}}
\label{sec:app_web_admin}
This is a web interface that allows one to manage the various parts of the \giraf system. This interface is for convenience for guardians and administrators when manipulating many items in the system, this interface supports all the features that exist in the tablet applications and more.