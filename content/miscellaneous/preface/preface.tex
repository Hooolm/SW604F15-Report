%!TEX root = ../../../super_main.tex

\chapter*{Preface}

This report is developed by project group SW604F15 from the Software programme at Aalborg University as part of the sixth semester project from February 2015 to June 2015. This project, commonly called \giraf, is concerned with the continued development of a software suite, for Android tablets, intended to assist citizens diagnosed with an \emph{Autism Spectrum Disorder}. This software suite is based on software development from previous $6^{th}$ semester students at Aalborg University, and has been continuously developed by the group alongside the rest of the students on this semester. The Giraf system has been in development since the spring of 2011.
\\\\
During the project period, the study method \emph{Problem-based learning}, also known as \emph{The AAU model} on Aalborg University has been utilized. However, this report is not structured like an ordinary report from the computer science department of Aalborg University. This report focuses on describing the process of developing on an existing code base and working intertwined with the more than 50 students in a software development context. Furthermore the report will also contain information useful for future project groups whom need to continue on the work.

% Reading guide
%!TEX root = ../../../super_main.tex

\section*{Reading Guide}
The report is splint into four parts, one part for each sprint in development process, and have been written in chronological order regarding this process. 

\subsection*{Translations}
\label{sub:translations}
Since the report is written in English and the GUI of the project is Danish some elements might be translated for istance 
\translated{Klik for at vælge indstillinger for hjemmeskærmen}{Click to choose settings for the home screen}. These translations is not documented anywhere other than in this report.
