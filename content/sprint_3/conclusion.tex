%!TEX root = ../../super_main.tex

\chapter{Sprint Conclusion}
\label{cha:conclusion_sprint_3}

\todo[inline]{NIKLAS SPRINTER FIX START - Uddyb lidt mere og lav nogle linjeskift if possibru}
The third sprint concluded with a close to finished \ct and a finished components for different dialogs. We have also implemented and modified a library called ShowcaseView in the \ct in order to guide users through the application and we have made the modified library available on the \giraf Git server. We estimate that the remaining work on the \ct will be easily done in the fourth sprint and we would therefore need additional work to do in the fourth sprint.
\\\\
We had some trouble with memory issues because of the hard memory limits set by the Android system and because of the many bitmaps used to display all the different pictograms. We came up with a temporary solution to the memory issues, until the database groups fixed the database abstraction classes, along with a nice way for other groups and future groups to use our workaround. This was implemented using the \androidinline{GirafPictogramItemView} class, which should be used along with an Android \androidinline{Adapter} when displaying multiple bitmaps.
\\\\
We have also been working along with different groups and helped them to improve their applications. 

\todo[inline]{NIKLAS SPRINTER FIX END}