%!TEX root = ../../../super_main.tex

\section{Using Pictosearch}
\label{sec:using_pictosearch}

\todo[inline]{Do we want a command for the name of Pictosearch?}

When using the \ct a common use case is that the user wants to find and use pictograms in various situations, for instance as an icon for a category creating categories or populating categories. For this sake we utilize an application developed for the \giraf software suite called Pictosearch. This application includes an \androidinline{Activity} subclass called PictoAdminMain \todo{Check if its stilled called this} which is able to return one or more pictograms to the calling \androidinline{Activity} when a user enters a search string and selects pictograms. 
\\\\
Pictosearch is used in three ways in the \ct. When creating a category, when editing a category, and when populating the category with pictograms. In order to use an \androidinline{Activity} from another application for this kind of request we utilize the \androidinline{Intent}-class. An \androidinline{Activity} is, using an \androidinline{Intent}-class instance, able to make a request to another activity asking for some result cross the operating system, in our case a very specific request to Pictosearch asking for ids of pictograms. 

\lstinputlisting[
    style = Java,
    caption = {A request for a single pictogram.},
    label = {lst:intent_example},
]{content/sprint_3/issues_solutions/intent_example.java}

The code in \lstref{lst:intent_example} shows an example of how we, using an \androidinline{Intent} instance, request a single pictogram through Pictosearch. We are able to ask Pictosearch for a single pictogram using the \androidinline{PICTO_SEARCH_SINGLE_TAG} tag. This opens the main activity of Pictosearch in a state where where the user is able to pick, at most, one pictogram. This is used when picking an icon for category, whereas \androidinline{PICTO_SEARCH_MULTI_TAG} is used when populating a category with pictograms. 
\\\\
When a user has selected the desired pictograms he is returned to the \ct, where the method \androidinline{onActivityResult} is called with the result. Using the request code (\androidinline{GET_SINGLE_PICTOGRAM} as in \lstref{lst:intent_example}), this method will get the same request code \androidinline{GET_SINGLE_PICTOGRAM} and it will there by know which request that has been answered and can now perform some action based on this, in this case we take the identifier for the pictogram and store it on the category it was picked for.

