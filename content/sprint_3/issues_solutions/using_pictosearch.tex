%!TEX root = ../../../super_main.tex

\section{Using Pictosearch}
\label{sec:using_pictosearch}

When using the \ct, a common use case is that the user wants to find and use pictograms in various situations, for instance as an icon for a category or populating categories. For this sake, we utilize an application developed for the \giraf software suite called \ps. This application includes an \androidinline{Activity} subclass called PictoAdminMain \todo{Check if its stilled called this} which is able to return one or more pictograms to the calling \androidinline{Activity} when a user enters a search string and selects pictograms. 
\\\\
\ps is used in three ways in the \ct: When creating a category, when editing a category, and when populating the category with pictograms. In order to use a remote \androidinline{Activity} from another application for this kind of request we utilize the \androidinline{Intent}-class. An \androidinline{Activity} is, using an \androidinline{Intent}-class instance, able to make a request to another activity asking for some result cross the operating system, in our case a very specific request to \ps asking for IDs of pictograms. 

\lstinputlisting[
    style = Java,
    caption = {A request for a single pictogram.},
    label = {lst:intent_example},
]{content/sprint_3/issues_solutions/intent_example.java}

The code in \lstref{lst:intent_example} shows an example of how we, using an \androidinline{Intent} instance, open \ps and ask for a single pictogram. The thing to notice is that we are able to ask \ps for a single pictogram using the \androidinline{PICTO_SEARCH_SINGLE_TAG}, where the user is only able to select a single pictogram. This is used when picking an icon for category, whereas \androidinline{PICTO_SEARCH_MULTI_TAG} is used when populating a category. 
\\\\
When a user has selected the desired pictogram(s), he is returned to the \ct, where an event is raised causing the method \androidinline{onActivityResult} to be called. Using the request code (\androidinline{GET_SINGLE_PICTOGRAM} in \lstref{lst:intent_example}), this method knows what request has been answered and can now perform some action based on this, which in this case is to store the identifier for the pictogram in the category it was picked for.

