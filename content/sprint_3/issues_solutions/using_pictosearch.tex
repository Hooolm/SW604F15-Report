%!TEX root = ../../../super_main.tex

\section{Using Pictosearch}
\label{sec:using_pictosearch}

\todo[inline]{Do we want a command for the name of Pictosearch?}

When using the \ct a common use case is that the user wants to find and use pictograms in various situations, for instance creating categories or populating categories. For this sake we utilize a service application developed for the \giraf software suite called Pictosearch. This application is able to fetch pictograms given a string. 
\\\\
In \ct this application has been used in three cases: When creating a category, when editing a category, and when populating a category with pictograms. In order to use another application for this kind of service we utilize the \androidinline{Intent}-class. This class is able to broadcast a request across the operating system, in our case a very specific request to Pictosearch asking for IDs of pictograms. 

\lstinputlisting[
    style = Java,
    caption = {A request for a single pictogram.},
    label = {lst:intent_example},
]{content/sprint_3/issues_solutions/intent_example.java}

The code in \lstref{lst:intent_example} shows an example of how we, using an \androidinline{Intent}, open Pictosearch and ask for a single pictogram. The thing to notice is that we are able to ask Pictosearch for a single pictogram using the \androidinline{PICTO_SEARCH_SINGLE_TAG}. This opens Pictosearch in a state where we at most is able to pick one pictogram. This is used when picking an icon for category, whereas \androidinline{PICTO_SEARCH_MULTI_TAG} is used when populating a category. 
\\\\
When a user has selected the desired pictogram(s) he is returned to the \ct, where an event is raised causing the method \androidinline{onActivityResult} to be called. Using the request code (\androidinline{GET_SINGLE_PICTOGRAM} in \lstref{lst:intent_example}), this method knows what request has been answered and can now perform some action based on this, which in this case is to store the identifier for the pictogram in the category it was picked for.

