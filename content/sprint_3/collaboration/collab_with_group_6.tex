%!TEX root = ../../../super_main.tex
\section{Collaboration with Group SW604F15}
\label{sec:collaboration_with_group_sw604f15}

After a general GUI design meeting, a new sequence deletion method was suggested by the customers. Sequences are used in some of the applications that are being developed by the other GUI groups. Sequences are lists of pictograms that should be viewed in chronological order, when then represent a behavior/action sequence which the citizen can execute. This could for example be the order in which a citizen should dress, e.g. putting on his socks before shoes. Previously when deleting sequences, the user would click a button to enter a ``Delete mode'' where, instead of clicking sequences to view them, clicks would instead delete a sequence. The customers requested the option to long-press a sequence to enter a ``MultiSelect mode'' where they were able to select multiple sequences for deletion and then afterwards press the trashcan icon. When a sequence is long pressed, the main menu is supposed to enter this mode and a trashcan button should appear in the top-bar. It should be possible to select and deselect sequences through regular press once inside the selection mode. It should furthermore be possible to deselect and exit the selection mode upon pressing the back button in the action bar.\\

In our project, we had need for a feature just like this one, which instead was focused around selecting categories instead of sequences. To quickly overcome this task, we grouped up with SW606f15, in order to resolve their issue first so we could implement the same structure. The difference between the two projects is that our solution should only select a single item with a regular press, where theirs should be able to select multiple when initiated by a long-press. \\
 
Previously, their project was based around the use of sequence objects and their respective ids to create views that represent the sequences. This resulted in not being able to find the view that belonged to the individual sequence, without passing both sequence and its view when using them in a method. This made it impossible to create the deselection feature when pressing the back button, since the back button was overriding the original \androidinline{onBackPressed} method, which does not take any parameters. \\
 
Because of this we decided to implement a new data structure for their project which contains both a sequence and its respective view. Through this, it is possible to create a list of the sequences selected during the selection mode. This allows us to look through the selected sequences list and access their views directly, so that we can modify them as deselected once the back button is pressed. \\
 
We furthermore implemented this using adapters as previously described, to improve performance when iterating through the list of sequences. This method of implementation could be directly transferred into our own project with minimal changes. \\