%!TEX root = ../../../super_main.tex
\section{Fragment Manager}
\label{sec:fragment_manager}

When nesting Android \androidinline{Fragment} objects, i.e having a \androidinline{Fragment} which contains another \androidinline{Fragment} in its layout, it is crucial that parent \androidinline{Fragment} objects uses a \androidinline{FragmentManager} provided by a call to \androidinline{getChildFragmentManager} instead of the \androidinline{FragmentManager} provided by the parent activity. The \giraf-class \androidinline{AppManagementFragment} failed to do this. Failing to use child-\androidinline{FragmentManager} results in exceptions in new versions of the Android API, and therefore it has to be solved. 
\\\\

\todo{DONE-start(holm) - Describe how we tested that this works. (Old tablet with API 15, tablet with API 17 and emulator with API 20)}
Since the \giraf system supports API levels 15, 17, and 19, this issue has to be solved differently based on the used device's API level. API level 15 does not support child-\androidinline{Fragment}, and therefore does not support child-\androidinline{FragmentManager} either. The solution to the issue is therefore to check which API level is present on the device, and if the device is at level 17 or above, use the child-\androidinline{FragmentManager}, and otherwise use the regular \androidinline{FragmentManager}. This will be implemented where needed in the \giraf system and resolve the exceptions that are currently present. This solution was verified by testing the solution on an elder tablet using API level 15, a newer tablet using API level 17 and an emulator using API level 20. These tests showed no conflicts regarding this issue.
\todo{DONE-end(holm)}