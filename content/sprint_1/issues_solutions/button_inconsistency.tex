%!TEX root = ../../../super_main.tex
\section{Button Inconsistency}
\label{sub:button_inconsistency}

The design of the GUI was found to be inconsistent in general around the \giraf-software suite. This issue was discovered by the \emph{SW606F15} group, who is the group responsible for user requirements and the graphical design guide of the project. In cooperation with this group, it was found that the existing implementations of the GUI-components needed rework, because of the code quality and structure. \todo{Fortæl at det findes i griafcomponents.} One of the problems found by \emph{SW606F15} is illustrated in \figref{fig:button_inconsistency} where it can be seen that the settings button (top most button) and the logout button (lower most button) are inconsistent in the graphical design.

\begin{figure}[!htbp]
    \centering
    \includegraphics[width=0.1\textwidth]{sprint_one/enforce_design_guide_in_settings/button_inconsistency}
    \caption{Example of inconsistent Buttons}
    \label{fig:button_inconsistency}
\end{figure}

This motivated \emph{SW606F15} to create a graphical design document which will be used to enforce a standard for the GUI, and which can be found in their report\footnote{This report has not yet been published, and we are therefore unable to cite it properly}. Furthermore, this design flaw was found in the GUI of the \launcher, which made us commit to the job of solving this problem.
\\\\
To solve the button inconsistency, a standardized button component should be developed. This component should correspond to the looks and feel described in the design document created by group \emph{SW606F15}. At first, the design document indicated that there should exist different button types. Each type should have a specific background color associated with it. For instance, buttons with user-actions should have a brown background as seen in \figref{fig:ugly_and_beautiful_buttons_example_one}. However, after implementing this feature, it was discovered that the users at \emph{Birken} did not like these new buttons, and would rather keep the look of the old buttons as seen in \figref{fig:ugly_and_beautiful_buttons_example_two}. 


\todo{Henvis til figuren og sig at de her på figuren er brugt i en sidebare som ofte er en usecase.}
\begin{figure}[!htbp]
    \centering

    \begin{subfigure}[t]{0.3\textwidth}
    	\centering
        \includegraphics[width=0.3\textwidth]{sprint_one/ugly_and_beautiful_buttons/button_redesign_beautiful}
        \caption{Colorful buttons}
        \label{fig:ugly_and_beautiful_buttons_example_one}
    \end{subfigure}
    \hspace{5em} 
    \begin{subfigure}[t]{0.3\textwidth}
    	\centering
        \includegraphics[width=0.3\textwidth]{sprint_one/ugly_and_beautiful_buttons/button_redesign_ugly}
        \caption{Yellow buttons}
        \label{fig:ugly_and_beautiful_buttons_example_two}
    \end{subfigure}
    
    \caption{Solution to inconsistent button design}
    \label{fig:ugly_and_beautiful_buttons_example_solution}
\end{figure}

To create the standardized button, a set of icons were made by group \emph{SW606F15}. These icons were then combined with a rounded square with a gradient background with the colors corresponding to the old buttons. All of this was made in the \gc-module, which means that all applications can utilize this new type of button. It is however still possible to use the old icons and deprecated buttons, but we encourage other groups to use these new components in their layouts.
\\\\
Buttons are not the only thing that is inconsistent in the design, however we did not have enough time to create any other standardized components this sprint.