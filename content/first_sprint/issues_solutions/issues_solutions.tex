%!TEX root = ../../../super_main.tex

\chapter{Issues and Solutions}

We conducted a series of unstructured tests and UI/Application Exerciser Monkey (Monkey) \parencite{android_monkey} tests in order to explore the launcher and find bugs during the start of the first sprint. 

Monkey is a tool, which comes with the Android Debug Bridge (adb) \parencite{android_adb}, used to perform series of pseudo-random streams of user events to test the application UI under stress.

We found the Monkey tool to be useful and indeed it did help us find a few concurrency issues when switching between tabs in the settings activity of the launcher.

Mistakes, in general, might occur multiple times in code base, and will be corrected when discovered, but will only be  mentioned further.



\subsection{Settings tabs crash}



Monkey test

LoadApplicationTask

AppViewCreationUtility.createAppImageView

synchronized

\subsection{Fragment management}

When nesting Android \androidinline{Fragment} objects, i.e having a \androidinline{Fragment} which contains another \androidinline{Fragment} in its layout, it is imperative that the parent Fragments uses a \androidinline{FragmentManager} provided by a call to \androidinline{getChildFragmentManager} instead of the \androidinline{FragmentManager} provided by the parent activity. The \giraf-class \androidinline{AppManagementFragment} failed to do this.   

AppManagementFragment




\subsection{Fragment constructors}

According to the Android API documentation \parencite{android_dev_fragment} Android Fragment subclasses classes must make a public no-argument constructor, a default constructor, available. However the \giraf-class \androidinline{SettingsLauncher} failed to provide such a default constructor which caused crashes. The standard practice when sub classing \androidinline{Fragment} objects is to provide a static factory method, which creates a new instance of the \androidinline{Fragment} subclass, called \androidinline{newInstance}. \androidinline{newInstance} takes its arguments and passes them to the created \androidinline{Fragment} subclass object through an Android \androidinline{Bundle} instance. 

%!TEX root = ../../../super_main.tex

\section{Inconsistent tab size in settings panel}

