%!TEX root = ../../../super_main.tex

\chapter{Issues and Solutions}

We conducted a series of unstructured tests and UI/Application Exerciser Monkey (Monkey) \parencite{android_monkey} tests in order to explore the launcher and find bugs during the start of the first sprint. 

Monkey is a tool, which comes with the Android Debug Bridge (adb) \parencite{android_adb}, used to perform series of pseudo-random streams of user events to test the application UI under stress.

We found the Monkey tool to be useful and indeed it did help us find a few concurrency issues when switching between tabs in the settings activity of the launcher.

<<<<<<< HEAD:content/first_sprint/issues_solutions.tex
Mistakes, in general, might occur multiple times in code base, and will be corrected when discovered, but will only be  mentioned further.



\subsection{Settings tabs crash}



Monkey test

LoadApplicationTask

AppViewCreationUtility.createAppImageView

synchronized

\subsection{Fragment management}

When nesting Android \javainline{Fragment} objects, i.e having a \javainline{Fragment} which contains another \javainline{Fragment} in its layout, it is imperative that the parent Fragments uses a \javainline{FragmentManager} provided by a call to \javainline{getChildFragmentManager} instead of the \javainline{FragmentManager} provided by the parent activity. The \giraf-class \javainline{AppManagementFragment} failed to do this.   

AppManagementFragment




\subsection{Fragment constructors}

According to the Android API documentation \parentcite{android_dev_fragment} Android Fragment subclasses classes must make a public no-argument constructor, a default constructor, available. However the \giraf-class \javainline{SettingsLauncher} failed to provide such a default constructor which caused crashes. The standard practice when sub classing \javainline{Fragment} objects is to provide a static factory method, which creates a new instance of the \javainline{Fragment} subclass, called \javainline{newInstance}. \javainline{newInstance} takes its arguments and passes them to the created \javainline{Fragment} subclass object through an Android \javainline{Bundle} instance. 
=======
%!TEX root = ../../../super_main.tex

\section{Inconsistent tab size in settings panel}
>>>>>>> origin/master:content/first_sprint/issues_solutions/issues_solutions.tex
