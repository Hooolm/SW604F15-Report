%!TEX root = ../../../super_main.tex
\subsection{GirafActivity and ActionBar}
\label{sec:giraf_activity_actionbar}

One of the user stories was to get a top-bar, called \androidinline{ActionBar} in Android, in all applications with a back-button for navigation between applications and their activities. A developerstory, which said that there should be build a common \androidinline{ActionBar} for use across all sub-projects, was created on the basis of this user story. The implementation of the new \giraf \androidinline{ActionBar} is described in this section. A shortened version of this section was published on the \giraf Redmine \parencite{redmine} forum.

\subsubsection{Idea and Implementation}

The idea of the new \androidinline{ActionBar} is that it should include a centered title of the current activity and two dynamic rows of \giraf buttons, in which the client of the \androidinline{ActionBar} should be able to add buttons, while always having a back-navigation-button in the upper leftmost corner. This back-navigation-button should have the same behavior as the Android ``hardware-back-button''.

The new \androidinline{ActionBar} was implemented by subclassing the Android \androidinline{Activity}, with a class called \androidinline{GirafActivity}, and by creating two new Android application themes called \androidinline{GirafTheme} and \androidinline{GirafTheme.NoTitleBar}. The idea is then that all activities across all projects should subclass \androidinline{GirafActivity} and thereby get the customizations of the \androidinline{ActionBar}. Clients of \androidinline{GirafActivity} should then for each \androidinline{GirafActivity}, in their \androidinline{AndroidManifest.xml} file, specify if they want an \androidinline{ActionBar} or not by specifying either \androidinline{GirafTheme} or \androidinline{GirafTheme.NoTitleBar} as their theme.

%  Snak om linearlayouts
The custom \androidinline{ActionBar} is implemented using a custom layout, an Android \androidinline{RelativeLayout}, for the \androidinline{ActionBar}. The \androidinline{RelativeLayout} then contains a centered \androidinline{TextView} for the title and a back-\androidinline{GirafButton} placed at the leftmost position in the \androidinline{ActionBar}. A \androidinline{LinearLayout} is then placed to the right of the back-button and another \androidinline{LinearLayout} is placed to the right of the title; both with the purpose of containing \androidinline{GirafButton} instances added by the client of the \androidinline{GirafActivity} i.e. the client of the customized \androidinline{ActionBar}. 

\subsubsection{GirafActivity API}

The \androidinline{GirafActivity} provides an easy API for adding \giraf buttons, i.e. instances of the \androidinline{GirafButton} class, to its customized \androidinline{ActionBar} through a method called \androidinline{addGirafButtonToActionBar}. \androidinline{addGirafButtonToActionBar} takes two arguments; one being a \androidinline{GirafButton} and the other a flag which is either \androidinline{GirafActivity.LEFT} or \androidinline{GirafActivity.RIGHT}. The \androidinline{GirafButton} is then added to the leftmost position of the title, and to the right of the back-button, if the \androidinline{GirafActivity.LEFT} flag is used or conversely to the right of title if the \androidinline{GirafActivity.RIGHT} flag is used. 
The default behavoir of the back-navigation-button in the \androidinline{ActionBar}, which is to \androidinline{finish} the current \androidinline{Activity}, can be changed by overriding the \androidinline{onBackPressed} method of the corresponding \androidinline{GirafActivity}.          
