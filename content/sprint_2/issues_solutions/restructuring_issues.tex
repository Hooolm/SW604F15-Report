%!TEX root = ../../../super_main.tex
\section{Restructuring Issues}
\label{sec:restructuring_issues}

During the first sprint, a lot of changes to the Git submodule structure were introduced by a \emph{Build and Deployment} group of the multiproject. These changes include changing a lot of submodule dependencies, changing namespaces for submodules, and changing library dependencies. The library dependencies are handled using an open source build automation tool called Gradle \parencite{gradle}, which is a standard build script in the Android Studio IDE. After the aforementioned changes had been implemented for the submodule structure, it had not been enforced in all of the \giraf projects who implemented the submodules. Because of this, many unforeseen issues had been introduced to the system, hereunder mismatches between project structure, \mono{gradle.build} files and \mono{gradle.settings} files, and furthermore git issues related to referring to old versions of submodules. Partially due to these changes, the \ct was unable to build and run on any Android device, and we therefore had to spend a lot of time and resources on addressing these issues. 
\\\\
After fixing all the problems related to the \ct, we talked to the other groups of the GUI subproject, who told us that some of them have parts of the same issues that we have already found solutions for. We therefore decided to dispatch one member of our group to the other groups, such that they would also be able to progress with their development. This of course results in reduced resources for our own group for a period of time, which most likely results in reduced productivity at the end of the sprint, but we weigh the general progress of the multiproject higher than our own progress.
\\\\
Later in the sprint, it was discovered that some of the changes we had to make to the submodule structure of the project, in order to make the \ct compile and work, were actually something that one of the database groups had been working on. We therefore worked together with this database group in order to restore the project to a working state where their necessary components were also present. Even later, this change also proved futile, since the \emph{Build and Deployment} sub-project had been working on making the confusing submodule structure into online libraries instead. 
\\\\
Retrospectively, we should probably have sought assistance with one of the \emph{Build and Deployment} groups, since they probably could have fixed the problem faster than we managed to. However, we do not feel like it has been a complete waste of time, since we have learned a lot from the experience.