%!TEX root = ../../../super_main.tex
\subsection{GirafActivity and ActionBar}
\label{sec:giraf_activity_actionbar}

One of the user stories was to get a top-bar, called \androidinline{ActionBar} in Android (also known as \androidinline{ToolBar} in newer API levels), in all applications with a back-button for navigation between applications and their activities. A developer story, which said that a common action bar should be developed for use across all sub-projects, was created on the basis of this user story. The implementation of the new \giraf action bar is described in this section. A shortened version of this section was published on the \giraf Redmine forum \parencite{redmine}.

\subsubsection{Idea and Implementation}

The idea of the new action bar is that it should include a centered title of the current activity and two dynamic rows of \giraf buttons, in which the developers using the action bar should be able to add buttons, while always having a back-navigation-button in the upper leftmost corner. This back-navigation-button should have the same behavior as the Android software or hardware back button.
\\\\
The new action bar was implemented by subclassing the Android \androidinline{Activity}, with a class called \androidinline{GirafActivity}, and by creating two new Android application themes called \androidinline{GirafTheme} and \androidinline{GirafTheme.NoTitleBar}. The idea is then that all activities across all projects should subclass \androidinline{GirafActivity} and thereby get the customizations of the action bar. Developers using \androidinline{GirafActivity} should then for each \androidinline{GirafActivity}, in their \androidinline{AndroidManifest.xml} file, specify if they want an action bar or not by specifying either \androidinline{GirafTheme} or \androidinline{GirafTheme.NoTitleBar} as their theme.
\\\\
%  Snak om linearlayouts
The custom action bar is implemented using a custom layout, an Android \androidinline{RelativeLayout}, for the action bar. The \androidinline{RelativeLayout} then contains a centered \androidinline{TextView} for the title and a back-\androidinline{GirafButton} placed at the leftmost position in the action bar. A \androidinline{LinearLayout} is then placed to the right of the back-button and another \androidinline{LinearLayout} is placed to the right of the title; both with the purpose of containing \androidinline{GirafButton} instances. An example of the action bar can be seen in \figref{fig:topbar}.

\begin{figure}[!htbp]
    \centering
    \includegraphics[width=0.75\textwidth]{sprint_two/topbar}
    \caption{Actionbar of a \giraf Activity}
    \label{fig:topbar}
\end{figure}         