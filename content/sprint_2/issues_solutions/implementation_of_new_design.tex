%!TEX root = ../../../super_main.tex
\section{Implementation of New Design}
\label{sec:implementation_of_new_design}

The implementation of the design from \secref{sec:improved_design} has has been partially completed during the second sprint. The main goal of second sprint in regards to the \ct was to implement the part of the tool, where guardians are able to create categories for an entire institution, and later copy categories to citizens.
\\\\
As mentioned previously, the user is greeted by a homescreen where they can see their current categories or add new ones. This can be seen in \figref{fig:ct_home_screen}. If the user presses the button labeled \translated{Opret ny kategori}{Create new category}, the user is able to create a new category which will be placed in the list on the left. 

\begin{figure}[!htbp]
    \centering
    \includegraphics[width=0.75\textwidth]{sprint_two/implementation_of_new_design/homescreen}
    \caption{\ct home screen}
    \label{fig:ct_home_screen}
\end{figure}

If the user selects a category, the pictogram list for that category will be shown, which can be seen in \figref{fig:ct_category_view}.\\

\begin{figure}[!htbp]
    \centering
    \includegraphics[width=0.75\textwidth]{sprint_two/implementation_of_new_design/categoryview}
    \caption{\ct category selected screen}
    \label{fig:ct_category_view}
\end{figure}

Both the list of categories as well as the scrollable grid related to the category have been implemented using subclasses of \androidinline{AdapterView} and \androidinline{Adapter}, which makes sure that items in the list are only loaded when the users scroll past it. This makes sure that excessive amounts of memory will not be used to store the complete list of views representing categories or pictograms, and thereby make sure that the application does not crash due to overflow of views in the grid.