%!TEX root = ../../super_main.tex

% Packages used
\usepackage{listings}

% ================================================ %

\captionsetup[lstlisting]{
    format = listing
}

% Default style for all lstlistings
\lstset 
{
    backgroundcolor = \color{white},
    keywordstyle = \color{blue},
    commentstyle = \color{gray!75}\textit,
    stringstyle = \color{green},
    basicstyle = \scriptsize\ttfamily,
    numberstyle = \tiny,
    numbers = left,
    breaklines = true,
    breakatwhitespace=true,
    showstringspaces = false,
    tabsize = 3,
    captionpos = t,
    extendedchars = true,
    escapeinside = {//*}{\^^M}, % Use latex inside lstlistings. For instance for refferences.
    frame = tblr,
    backgroundcolor = \color{gray!5},
    xleftmargin = 3.5pt,
}

% Write "Code snippet" instead of "listing".
\renewcommand{\lstlistingname}{Code Snippet}

% General style for the whole lstlisting
\DeclareCaptionFont{white}{\color{white}}
\DeclareCaptionFormat{listing}{\colorbox{gray}{\parbox{0.9934\textwidth}{#1#2#3}}}
\captionsetup[lstlisting]{format = listing, labelfont = white, textfont = white}

\lstdefinelanguage{TAInC} 
{
    morekeywords=[1]{number, boolean}, 
    keywordstyle=[1]\color{NavyBlue},
    morekeywords=[2]{do, if, else, run, when, startup, return, NAME},
    keywordstyle=[2]\color{Magenta},
    morekeywords=[3]{Tank, Gun, Battle},
    keywordstyle=[3]\color{Green},
    morekeywords=[4]{BattleEnded,BulletHit, BulletHitBullet, BulletMissed, Death, HitByBullet, HitTank, HitWall, TankDeath, RoundEnded, ScannedTank, Win},
    keywordstyle=[4]\color{Bittersweet},
    morekeywords=[5]{action, calculation},
    keywordstyle=[5]\color{Fuchsia},
    sensitive = true,
    morecomment = [l]{//},
    morecomment = [n]{/*}{*/}
}

% Custom lststyle named tainc
\lstdefinestyle{tainc}
{
    breaklines = true,
    columns = fullflexible,
    language = TAInC
}

% Custom lstinline for TAInC named taincinline
\newcommand{\taincinline}[1]{\lstinline[style = tainc, basicstyle = \ttfamily\normalsize]{#1}}

% Import the Java language
\lstloadlanguages{Java}

% Custom lststyle named java
\lstdefinestyle{java}
{
    breaklines = true,
    language = Java,
    columns = fullflexible,
    stringstyle=\color{eclipse_blue},
    morekeywords=[1]{class, return}, 
    keywordstyle=[1]\color{eclipse_red}
}

% Custom lstinline for Java named javainline
\newcommand{\androidinline}[1]{\lstinline[style = java, basicstyle = \ttfamily\normalsize]{#1}}

% Custom highlightning for Xtext
\lstdefinelanguage{Xtext} 
{
    morekeywords=[1]{returns, current, terminal}, 
    keywordstyle=[1]\color{DarkOrchid},
    morekeywords=[2]{startup, run, do, action},
    keywordstyle=[2]\color{eclipse_blue},
    morekeywords=[3]{name},
    keywordstyle=[3]\color{red},
    stringstyle=\color{Goldenrod},
    sensitive = true,
    morecomment = [l]{//},
    morecomment = [n]{/*}{*/}
}

% Custom lststyle named xtext
\lstdefinestyle{xtext}
{
  breaklines = true,
  language = Xtext,
  columns = fullflexible
}

% Custom lstinline for Xtext named xtextinline
\newcommand{\xtextinline}[1]{\lstinline[style = xtext, basicstyle = \ttfamily\normalsize]{#1}}

\lstdefinelanguage{Xtend} 
{
    morekeywords=[1]{class, extends, new, extension, import, package, def, if, else, val, var, instanceof, null, true, false, return, default, private, case, switch, dispatch}, 
    keywordstyle=[1]\color{DarkOrchid},
    stringstyle=\color{eclipse_blue},
    sensitive = true,
    morecomment = [l]{//},
    morecomment = [n]{/*}{*/}
}

\lstdefinestyle{xtend}
{
    breaklines = true,
    columns = fullflexible,
    language = Xtend,
    stringstyle=\color{eclipse_blue},
    showstringspaces=false
}

% Custom lstinline for Xtend named xtendinline
\newcommand{\xtendinline}[1]{\lstinline[style = xtend, basicstyle = \ttfamily\normalsize]{#1}}

% Import the C language
\lstloadlanguages{C}

% Custom lststyle named c
\lstdefinestyle{c}
{
    breaklines = true,
    language = C,
    columns = fullflexible,
    stringstyle=\color{eclipse_blue},
    morekeywords=[1]{return}, 
    keywordstyle=[1]\color{eclipse_red},
}

% Custom lstinline for C named cinline
\newcommand{\cinline}[1]{\lstinline[style = c, basicstyle = \ttfamily\normalsize]{#1}}

% Import the C# language
\lstloadlanguages{[Sharp]C}

% Custom lststyle named cs
\lstdefinestyle{cs}
{
    breaklines = true,
    language = [Sharp]C,
    columns = fullflexible,
    stringstyle=\color{eclipse_blue},
    morekeywords=[1]{return}, 
    keywordstyle=[1]\color{eclipse_red},
}

% Custom lstinline for C# named csinline
\newcommand{\csinline}[1]{\lstinline[style = cs, basicstyle = \ttfamily\normalsize]{#1}}

% Workaround on global float for lstlisting
% \lstset{float}
% \makeatletter
% \let\lst@floatdefault\lst@float
% \makeatother